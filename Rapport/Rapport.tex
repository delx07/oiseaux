\documentclass[9pt,twocolumn,twoside,]{pnas-new}

% Use the lineno option to display guide line numbers if required.
% Note that the use of elements such as single-column equations
% may affect the guide line number alignment.


\usepackage[T1]{fontenc}
\usepackage[utf8]{inputenc}

% tightlist command for lists without linebreak
\providecommand{\tightlist}{%
  \setlength{\itemsep}{0pt}\setlength{\parskip}{0pt}}


% Pandoc citation processing
\newlength{\cslhangindent}
\setlength{\cslhangindent}{1.5em}
\newlength{\csllabelwidth}
\setlength{\csllabelwidth}{3em}
\newlength{\cslentryspacingunit} % times entry-spacing
\setlength{\cslentryspacingunit}{\parskip}
% for Pandoc 2.8 to 2.10.1
\newenvironment{cslreferences}%
  {}%
  {\par}
% For Pandoc 2.11+
\newenvironment{CSLReferences}[2] % #1 hanging-ident, #2 entry spacing
 {% don't indent paragraphs
  \setlength{\parindent}{0pt}
  % turn on hanging indent if param 1 is 1
  \ifodd #1
  \let\oldpar\par
  \def\par{\hangindent=\cslhangindent\oldpar}
  \fi
  % set entry spacing
  \setlength{\parskip}{#2\cslentryspacingunit}
 }%
 {}
\usepackage{calc}
\newcommand{\CSLBlock}[1]{#1\hfill\break}
\newcommand{\CSLLeftMargin}[1]{\parbox[t]{\csllabelwidth}{#1}}
\newcommand{\CSLRightInline}[1]{\parbox[t]{\linewidth - \csllabelwidth}{#1}\break}
\newcommand{\CSLIndent}[1]{\hspace{\cslhangindent}#1}


\templatetype{pnasresearcharticle}  % Choose template

\title{Passereaux en passerelle}

\author[]{Thomas Cournoyer}
\author[]{Félix Richard}
\author[]{Liam Ryan}
\author[]{Xavier Delisle}

  \affil[]{Université de Sherbrooke, Cours BIO500}


% Please give the surname of the lead author for the running footer
\leadauthor{}

% Please add here a significance statement to explain the relevance of your work
\significancestatement{}


\authorcontributions{}



\correspondingauthor{\textsuperscript{} }

% Keywords are not mandatory, but authors are strongly encouraged to provide them. If provided, please include two to five keywords, separated by the pipe symbol, e.g:


\begin{abstract}
Résumé

Notre projet lui consistait à créer une banque de donnée d'observation
acoustique d'oiseaux pour différents sites d'observations au Québec. Une
fois la banque de données assembler nous voulions analyser la
répartition de passériformes en abondance selon la latitude et les mois
de l'année. Les résultats se sont révélés être en accordance avec ce
nous attendions observer avec une baisse des observations selon la
latitude ainsi qu'une variation dans les observations selon les périodes
migratoires. Certains résultats étaient plus ou moins inattendus comme
la différence en abondance journalière ainsi que la présence de
plusieurs observations en haute latitude. Il a été conclu qu'une
recherche plus approfondie pourrait être réalisée, mais cette fois avec
des points d'échantillonnages répartit sur l'entièreté du territoire
québécois.
\end{abstract}

\dates{This manuscript was compiled on \today}
\doi{\url{www.pnas.org/cgi/doi/10.1073/pnas.XXXXXXXXXX}}

\begin{document}

% Optional adjustment to line up main text (after abstract) of first page with line numbers, when using both lineno and twocolumn options.
% You should only change this length when you've finalised the article contents.
\verticaladjustment{-2pt}



\maketitle
\thispagestyle{firststyle}
\ifthenelse{\boolean{shortarticle}}{\ifthenelse{\boolean{singlecolumn}}{\abscontentformatted}{\abscontent}}{}

% If your first paragraph (i.e. with the \dropcap) contains a list environment (quote, quotation, theorem, definition, enumerate, itemize...), the line after the list may have some extra indentation. If this is the case, add \parshape=0 to the end of the list environment.

\acknow{}

\hypertarget{introduction}{%
\section*{Introduction}\label{introduction}}
\addcontentsline{toc}{section}{Introduction}

La migration des oiseaux a toujours été un sujet d'intérêt écologique au
Québec. Dans les dernières années, de plus en plus de recherches
utilisant des points d'écoute ont été réalisées et c'est ce type de
donnée que nous avons tenté d'utiliser afin de créer une banque de
données qui pourrait être utilisé à des fins de recherche. Nous avons
ciblé les passériformes comme sujet d'étude, car c'est le plus grand
ordre d'oiseau présent dans la banque de données. Dans un premier temps,
nous avons analysé le nombre d'espèces présentes selon la latitude, puis
l'abondance des passériformes selon la latitude et pour finir nous avons
effectué une analyse de l'abondance mensuelle moyenne des passériformes.

\hypertarget{muxe9thodes-et-ruxe9sultats}{%
\section*{Méthodes et résultats}\label{muxe9thodes-et-ruxe9sultats}}
\addcontentsline{toc}{section}{Méthodes et résultats}

Les données utilisées proviennent du suivi de la biodiversité acoustique
du ministère de l'Environnement, de la Lutte contre les Changements
Climatiques, de la Faune et des Parcs (MELCCFP). Ces données sont
obtenues par enregistrement des sons entendus à certains points où les
enregistreurs sont posés, puis analysés par des taxonomistes pour
identifier quelles espèces sont entendues. Le protocole est disponible
ici : {[}Protocole d'inventaire acoustique multiespèce avec appareil
Song Meter Mini Bat (SMMB){]}
(\url{https://mffp.gouv.qc.ca/documents/faune/protocole-inventaire-acoustique-multiespece.pdf}).
Les résultats obtenus sont quatre figures observant les abondances par
latitude des toutes les observations, l'abondance par latitude des
passériformes et l'abondance par mois des passériformes.

\hypertarget{submitting-manuscripts}{%
\section*{Discussion}\label{submitting-manuscripts}}
\addcontentsline{toc}{section}{Discussion}

\begin{figure}
\centering
\includegraphics[width=0.5\textwidth,height=0.4\textheight]{Figure2.png}
\caption{Diagramme en boîte à moustache de l'abondance journalière selon
la latitude \label{fig:plot1}}
\end{figure}

Les distributions par latitude de la figure \ref{fig:plot1} semblent
beaucoup varier, cela peut être par cause de différence d'effort
d'échantillonnage par latitude ou même par site. Une autre considération
est que les espèces migratrices ont peut-être des observations à 58,
ainsi que des observations répétées plus tard dans l'année entre 45 et
50. La distribution elle-même comporte des valeurs semblables de 45 à 48
de latitude et un gros pic à 58. On prédit qu'environ 75\% des espèces
situées à une latitude de 55 ou plus migreront vers le sud, comparé à
55-65\% des espèces qui sont à des latitudes de 45-50 (1). C'est
pourquoi nous pouvons être sûrs qu'il y a probablement de la lecture
double entre les espèces à 58 et ceux qui sont retrouvés plus bas.
L'autre graphique montre que la tendance de distribution est assez nulle
selon l'autre graphique de distribution qui n'a pas été simplifié par
unité de latitude ce qui signifie que la distribution des oiseaux est
assez uniforme et la latitude importe peu à leur présence. En fait, les
niches selon les latitudes sont remplies et il n'y a pas de pression
évolutive qui provient de ce facteur dans le jour présent (2), donc on
s'attend justement à une distribution uniforme. Le graphique atteint nos
attentes initiales, mais avec une latitude 58 qui a une étendue plus
grande que prévu, probablement due à l'effet de la migration.

\begin{figure}
\centering
\includegraphics[width=0.5\textwidth,height=0.4\textheight]{Figure3.png}
\caption{Graphique de l'abondance de passériformes selon la latitude
\label{fig:plot2}}
\end{figure}

Les résultats de la figure \ref{fig:plot2} nous démontrent une grande
abondance d'observation entre les latitudes 45 et 50, ainsi qu'une
présence modérée d'observation à proximité de la latitude 60. La
tendance générale de notre modèle semble démontrer une baisse des
observations plus les sites sont aux nord. Ces résultats ne sont pas
surprenants lorsqu'on considère que la majorité des passereaux sont des
oiseaux migrateurs qui pour la plupart préfèrent un climat plus tempéré
qui est typiquement plus présent dans le sud du Québec (3). La
distribution des oiseaux serait influencée autant par des facteurs
climatiques que des facteurs en lien avec les habitats suggérant que le
climat peut indirectement influencer la distribution des oiseaux en
affectant la végétation (3). Malgré cette tendance, nous avons un bon
nombre d'observations à proximité de la latitude 60, ce qui semble aller
à l'encontre de notre tendance. Il faut se rappeler que nos sites
d'échantillonnages sont répartis à différentes latitudes et que la
majorité de nos sites se situe dans le sud du Québec sauf pour quelques
sites répartis dans le nord du Québec à proximité de la latitude 60.
L'absence de site entre ces deux extrêmes viendrait expliquer pourquoi
il y a une chute si soudaine entre les abondances d'observations.
L'observation d'espèces de passereaux nordiques reste tout de même
conforme avec les résultats obtenus par (3) qui eux avaient pu observer
que certaines espèces restaient vraiment dans le nord de leur aire de
répartition. Il serait intéressant de refaire l'expérience avec plus de
sites qui seraient mieux répartis pour essayer de former un gradient
continu de site d'observation du sud au nord. Ce gradient permettrait
d'observer encore mieux la tendance, ou à l'inverse, de démontré que
cette tendance était seulement l'objet de l'écart entre la répartition
de nos sites d'observations.

\begin{figure}
\centering
\includegraphics[width=0.5\textwidth,height=0.4\textheight]{Figure4.png}
\caption{Diagramme à bandes de l'abondance mensuelle de passériformes
\label{fig:plot3}}
\end{figure}

La tendance d'abondance d'observations qu'on observe dans la figure
\ref{fig:plot3} suit bien ce à quoi on s'attendrait dans un contexte de
suivi d'oiseaux migrateurs selon les mois de l'année. On peut observer
une hausse au niveau des observations en mars, ce qui coïncide avec
l'arrivée des espèces migratrices au Québec (4, 5). Cette hausse est
aussi due au début de la période de nidification qui va se dérouler
jusqu'en mai (4, 5). Le deuxième pic d'observation est au mois d'août
qui est le début de la migration inverse des espèces qui vont
graduellement commencer leur migration au sud jusqu'au mois de novembre
où les dernières espèces migratrices quittent le Québec pour aller vers
des climats plus chauds (4, 5). Notre modèle représente donc bien les
variations attendues d'observations des oiseaux migrateurs selon les
différents mois de l'année au Québec.

\hypertarget{conclusion}{%
\section*{Conclusion}\label{conclusion}}
\addcontentsline{toc}{section}{Conclusion}

En conclusion, nos résultats de recherche sont pour la plupart ce à quoi
nous nous attendions. Un changement que nous pourrions apporter à cette
recherche serait d'utiliser plus de points d'écoute répartis
uniformément sur le territoire du Québec. Cette meilleure répartition
pourrait nous permettre d'obtenir de meilleures données et un meilleur
patron de détection et dispersion. Un projet de cette envergure répété
sur plusieurs années pourrait permettre de détecter l'impact qu'a les
changements climatiques sur la répartition et la dispersion des
passériformes au Québec, ainsi que de vérifier s'il y a des
modifications au niveau des évènements migratoires. Ce projet pourrait
permettre un meilleur suivi des populations d'oiseaux au Québec et
permettre de meilleures actions de conversation.

\begin{center}\rule{0.5\linewidth}{0.5pt}\end{center}

\showmatmethods

\hypertarget{references}{%
\section*{Références}\label{references}}
\addcontentsline{toc}{section}{Références}

\pnasbreak

\hypertarget{refs}{}
\begin{CSLReferences}{0}{0}
\leavevmode\vadjust pre{\hypertarget{ref-newton1996bird}{}}%
\CSLLeftMargin{1. }%
\CSLRightInline{Newton I, Dale L (1996) Bird migration at different
latitudes in eastern north america. \emph{The Auk} 113(3):626--635.}

\leavevmode\vadjust pre{\hypertarget{ref-rabosky2015minimal}{}}%
\CSLLeftMargin{2. }%
\CSLRightInline{Rabosky DL, Title PO, Huang H (2015) Minimal effects of
latitude on present-day speciation rates in new world birds.
\emph{Proceedings of the Royal Society B: Biological Sciences}
282(1809):20142889.}

\leavevmode\vadjust pre{\hypertarget{ref-desgranges2010potential}{}}%
\CSLLeftMargin{3. }%
\CSLRightInline{DesGranges J-L, Morneau F (2010) Potential sensitivity
of qu{é}bec's breeding birds to climate change sensibilit{é} potentielle
des oiseaux nicheurs du qu{é}bec aux changements climatiques.
\emph{Avian Conservation and Ecology} 5(2):5.}

\leavevmode\vadjust pre{\hypertarget{ref-gouvernement_canada}{}}%
\CSLLeftMargin{4. }%
\CSLRightInline{Gouvernement du canada (2023) Available at:
\url{https://www.canada.ca/fr/environnement-changement-climatique/services/prevention-effets-nefastes-oiseaux-migrateurs/periodes-generales-nidification/periodes-nidification.html}
{[}Accessed April 24, 2024{]}.}

\leavevmode\vadjust pre{\hypertarget{ref-saison_orinthologues}{}}%
\CSLLeftMargin{5. }%
\CSLRightInline{La saison préférée des ornithologues Available at:
\url{https://www.sepaq.com/blogue/saison-preferee-ornithologues.dot}
{[}Accessed April 24, 2024{]}.}

\end{CSLReferences}



% Bibliography
% \bibliography{pnas-sample}

\end{document}
